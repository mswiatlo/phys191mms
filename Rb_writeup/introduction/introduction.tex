%Introduction body
%Created MB 04-12

\section{Introduction}\label{introduction}

We focus on a specific case of optical pumping in which an ensemble of rubidium atoms is preferentiall pumped into a single magnetic sublevel using circularly polarized light. A number of quantum properties are observed when an ensemble of atoms are in a single spin projection state. The range of experiments available which utilize optical pumping, usually related to radiofrequency spectroscopy \cite{bloom}, are therefore very wide, with topics ranging from measurement of the g-factors of different atoms and the electron to analysis of the functional form of power-broadening and spin-alignment decay rates.

Our experiment consists of several measurements, most of which can be compared to analytically derived forms. First, general measurements of the effectiveness of the optical pumping, the properties of the transient behavior, and the behavior of Rabi oscillations are performed. By stepping across different radiofrequencies applied to the system and measuring the root mean square amplitude of the Rabi oscillations, we are able to find the frequency corresponding to the Zeeman energy splitting of the atoms and therefore calculate the atomic g-factors. Then, it is possible to measure the natural decay time $T_{1}$ of the metastable spin state by letting the system evolve without the presence of the optical pumping laser. Next, we measure the time constant $T_{2}$ associated with the natural spin decoherence rate of the spin projections by measuring the linewidth of the Rabi oscillation amplitudes as a function of frequency. Finally, we are able to measure the spin-exchange rate between $^{85}$Rb and $^{87}$Rb by analyzing the Rabi oscillations when pumping one isotope and driving the other with an oscillating magnetic field.

In Section~\ref{background}, \emph{Background}, we introduce the theoretical basis for optical pumping, the qualities of the rubidium atom which make it ideal for pumping, and the properties of Rabi oscillations. We derive the approximate analytic expressions for the power-broadening behavior of the linewidths $\gamma_{1}$ and $\gamma_{2}$ by solving a system of simplified 2 and 3 state rate equations. In Section~\ref{experimental}, \emph{Experimental}, we describe the experimental techniques, including the methods used to generate magnetic fields and the optical setup. Raw data accumulated from the experiments is presented in Section~\ref{results}, \emph{Results}. The algorithms used to process the data and the calculated values, as well as error analysis, are given in Section~\ref{analysis}, \emph{Analysis}. Finally, the results are summarized and compared to expected forms, and experimental improvements are suggested in Section~\ref{conclusion}, \emph{Conclusion}.