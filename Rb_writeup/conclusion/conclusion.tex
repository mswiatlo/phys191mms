%Conclusion body
%Created MB 04-12

\section{Conclusion}\label{conclusion}

In this experiment, we observe several properties of rubidium atoms in an optical pumping system. We observe Rabi oscillations by pumping atoms to a preferential state $m_{F}$ and coupling them with a radiofrequency signal to a lower $m_{F}$ level. The g-factors for both isotopes of rubidium are calculated to be $g_{F}=0.30\pm0.03$ for $^{85}$Rb and $g_{F}=0.46\pm0.05$ for $^{87}$Rb, within error of the respective expected values $g_{F}=0.33$ and $g_{F}=0.50$. The ratio of g-factors is calculated to be $g_{87}/g_{85}=1.5\pm0.2$, even in stronger agreement with the expected value of $g_{87}/g_{85}=1.5$. The optical pumping time for $^{85}$Rb in our experimental conditions was calculated to be $\tau = 5.9\pm 1.5$ ms and the spin relaxation time was found as $T_{1}=1.2 \pm 0.2$ ms, both in somewhat reasonable agreement with typical values \cite{vanier}, given our experimental differences. The spin decoherence times were calculated to be $T_2 = 86 \pm 12$ $\mu$s for $^{85}$Rb and $59 \pm 5$ $\mu$s for $^{87}$Rb; again, in somewhat reasonable agreement with typical values \cite{vanier}. Furthermore, the data collected fit very well to the power-broadening equations derived in Section SOMETHING. Finally, the spin exchange contribution to the spin decoherence times were measured $T^{se}_2=110 \pm 11$ $\mu$s for  $^{85}$Rb and $T^{se}_2 = 81 \pm 15$ $\mu$s for  $^{87}$Rb. Both are in good agreement with the theory in that they are larger than their respective $T_{2}$ and are therefore a large contribution to the entire spin decoherence time, as discussed in Section~\ref{spinexchange}.

There are a number of areas for potential improvement in the experiment. The resistive heater, for instance, corrupts measurements in the low radiofrequency amplitude measurements of $\gamma_{2}$. Other authors have suggested using a gas flame to overcome this limitation \cite{benumof}, but perhaps circulation of warm air heated by a source outside the mu-metal shielding would be more ideal. Furthermore, to reduce magnetic field inhomogeneities it would be useful to remove all magnetic materials from the inside of the shielding. A feedback circuit could be made to stabilize the current in the solenoid to reduce error in the calculated solenoid field; furthermore, some stabilization of the laser's current is also certainly possible. Improvements of this sort would allow for much more sensitive spectroscopic measurements: for example, using an ionizing electric discharge, it is possible to use the spin-exchange procedure to measure the g-factors of other gasses and even the electron.