%Introduction body
%Created MB 04-12

\section{Introduction}\label{introduction}

Superfluidity is the phenomenon in which a portion of atoms in a fluid form a single quantum degenerate state and acquire novel properties not seen in normal fluids. While both $^{4}$He and $^{3}$He undergo phase transitions to superfluids at low enough temperatures, we study the properties of $^{4}$He because its critical temperature is much higher. Above the critical temperature $T_{\lambda}$, $^{4}$He is referred to as helium I; below $T_{\lambda}$, it is referred to as helium II. The two-fluid model proposed by Landau \cite{landau} describes helium II as a mixture of normal helium and superfluid helium and explains many of the features observed.

The experiment consists of three main components. First, both the transition between helium I and helium II and the so-called `fountain effect' are observed. Both of these phenomena demonstrate the inability of helium II to support heat gradients. Second, we measure the speed of propagation of temperature waves, known as `second sound,'  as a function of temperature through helium II. Finally, we measure the specific heat of helium I and helium II in order to observe the $\lambda$-shaped discontinuity in the specific heat which gives the critical temperature its name.

In Section~\ref{background}, \emph{Background}, we introduce the theoretical basis for superfluidity and provide a theoretical understanding for a number of the phenomena observed in the experiment. The experimental methods, including temperature monitoring and control and heat generation, are described in Section~\ref{experimental}, \emph{Experimental}. Raw data accumulated from the experiments is presented in Section~\ref{results}, \emph{Results}. The algorithms used to process the data and the calculated values for the speed of second sound the specific heat, are given in Section~\ref{analysis}, \emph{Analysis}. Using Landau's two-fluid model, we are also able to calculate the fraction of normal and superfluid helium as a function of temperature. Finally, the results are summarized and experimental improvements are suggested in Section~\ref{conclusion}, \emph{Conclusion}.