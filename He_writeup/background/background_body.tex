%Background body
%Created MB 04-12

\section{Background}\label{background}

The $^4$He atom is one of the very simple and symmetrical systems
among the elements, with a filled innermost electron shell and no
overall electric or magnetic moment or angular momentum to the atom
\cite{atkins}. Due to the symmetric nature of helium atoms, the
interaction between them is very weak, and $^4$He liquifies at an
extremely low temperature of $4.21$ K. However, a more interesting
transition occurs at $2.17$ K, at which point liquid helium takes on
several unique properties. Below this temperature, termed the
$\lambda$ point, $^4$He\footnote{Although the isotope $^3$He is also
  capable of producing these effects, the transition temperature is
  much lower at $3\times 10^{-3}$ K and is unreachable with the
  cryogenic technology available to us. Thus, in the remainder of the
  paper, the isotope $^4$He is implied unless stated otherwise.}
acquires extremely high thermal conductivity, negligible viscocity,
and the ability to propagate temperature waves(second sound); in
addition, there is a $\lambda$-shaped discontinuity at the transition
in the heat capacity of liquid helium, which is how the $\lambda$
point acquires its name. To distinguish the two phases of liquid
helium, the liquid is referred to as helium I above the lambda point
and helium II below the lambda point (BLAHHH). The investigation of
the properties of helium II mentioned above will be the focus of this
paper.

\subsection{The Two-Fluid Model}


\begin{figure}[ht]
\begin{center}
\input{./figures/superfluid_fraction.png}
\caption{\small{}}
\label{figure:e_nu}
\end{center}
\end{figure}


\subsection{Second Sound}
\subsection{Heat Capacity}
