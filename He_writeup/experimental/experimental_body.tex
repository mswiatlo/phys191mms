%Experimental body
%Created MB 04-12

\section{Experimental}\label{experimental}

Several features are common to all three experiments performed. Two separate dewars are used in the experiment; both are Precision Cryogenic Systems model PVS-337 LHe Vapo-Shield Dewars. Aluminized Mylar superinsulation provides isolation to the external environment. The preparatory procedure for all experiments is to precool the chosen dewar to 77 K using liquid nitrogen and then to fill the dewar with liquid helium. Temperature control of the liquid helium is achieved by pumping on the liquid with a vacuum pump: changing the vapor pressure also changes the temperature in a predictable manner, subject to a calibration found in (something).

%note uniformity of temperature when pumping down

\subsection{Visual Observation Experiments}

The first experiment consisted of visual observations of different phenomena in superfluid helium. One helium dewar has a glass observation window near the bottom; while this increases coupling to the environment and thereby raises the lowest achievable temperature, it is still possible to pump the helium down to superfluid temperatures.

\subsubsection{Transition between He I and He II}

The most obvious phenomena is the transition between helium I and helium II. As the temperature is lowered through the lambda-point, it is possible to observe through the window the marked change in visual character of the liquid, which is a result of helium II's inability to support heat gradients and boiling.

\subsubsection{Superfluid fountain}

The next experiment allows for a particularly impressive method of displaying this inability to support heat gradients. A specially crafted glass container has a very narrow capillary tube on its top, and a wide semi-porous glass bottom. The semi-porous glass (made of very finely ground and compacted glass) allows only superfluid, not normal fluid, to pass. A 100 $\Omega$ resistor placed inside the container acts as a heater. When a voltage is applied to the resistor, the heated superfluid draws cooler superfluid to it in an effort to remove the temperature gradient; this sudden rush of superfluid forces some out through the capillary tube on top, creating a small helium fountain.

\subsection{Second Sound Experiment}

In the next experiment, we measure the propagation speed of second sound as a function of temperature.

Second sound is the propagation of heat in a superfluid as a wave, so a source of heat is necessary. A grid of Nichrome wire, with a resistance of about 10 $\Omega$ is used as a resistive heater. A HP3018A function generator creates an 80 $\mu$s low voltage (5 V) pulse, which is amplified to 40 V with a Kepco BOP 50-2M Power Amplifier and sent to the heater. 

To measure the actual propagation speed, a thermometer is also necessary. A carbon resistor (nominally 65 $\Omega$ at room temperature) is supplied with a constant 100 $\mu$A current, and the voltage across it is read by a voltmeter. A change in temperature of the resistor at very low temperatures corresponds to a change in resistance, and therefore a change in voltage. This resistor (referred to as a bolometer) is placed at the end of an adjustable sliding rod, whose distance from the Nichrome wire is able to be adjusted.

A heat pulse read on the bolometer is expected to be visible as a dip on an oscilloscope. As the temperature goes up from the pulse, the resistance goes down, and therefore the voltage goes down. The signal from the bolometer is very noisy and difficult to read, as the wires leading to it are unshielded so as to lower thermal contact into the liquid helium. A 60 Hz bandpass filter removes the worst source of noise; to remove other sources of noise and to amplify the signal, a EG\&G PARC 113 Pre-Amp is set to a gain of 10,000 with a 3 kHz high-pass filter and .1 Hz low-pass filter.

To perform measurements at a constant temperature, it is important to maintain the vapor pressure at a constant level. A series of valves attached to the vacuum pump allows for very fine control of the pumping rate, which can, with some persistence, be set at a level to temporarily balance out the heating from the outside environment. Changes in the volume of the liquid offset this delicate equilibrium, so it is necessary to constantly monitor the pressure and update the pumping rate to compensate.

To perform measurements of the velocity of second sound, first a constant temperature is selected. A series of heat pulses is sent using the function generator, and an oscilloscope averages over the response from the bolometer to further reduce noise. The time between the pulse from the generator and the first dip on the bolometer is recorded; this time corresponds to how long it took the pulse to travel from the nichrome wire to the bolometer. The distance between the wire and the bolometer is then altered and recorded, and the measurement repeated. Once 7 measurements are taken at one temperature, the temperature is changed and the process is repeated.

\subsection{Heat Capacity Experiment}

