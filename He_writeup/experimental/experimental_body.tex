%Experimental body
%Created MB 04-12

\section{Experimental}\label{experimental}

Our experiment constists of three main components: direct observation
of the visual properties of liquid helium, measurement of second sound
velocity as a function of temperature, and measurement of specific
heat as a function of temperature. Several features are common to all
three experiments performed. Two separate Precision Cryogenic Systems
model PVS-337 LHe Vapo-Shield Dewars are used for different stages fo
the experiment. Aluminized Mylar superinsulation provides isolation to
the external environment. The preparatory procedure for all
experiments is to precool the dewar to $77$ K using liquid nitrogen
and then to fill the dewar with liquid helium. Temperature control of
the liquid helium is achieved by pumping on the liquid with a
mechanical vacuum pump: changing the vapor pressure also changes the
temperature in a predictable manner, subject to a calibration found in
\ref{temperaturecalibration}. The temperature of helium I is uniform
throughout the bath if the system is being pumped down (that is, for
decreasing temperature); the temperature of helium II is uniform
constantly, as described in Section~\ref{thermalconductivity}.

\subsection{Visual Observation Experiments}

The first experiment consists of visual observations of superfluid
helium phenomena. One helium dewar has a glass observation window near
the bottom; while this increases coupling to the environment and
thereby raises the lowest achievable temperature, it is still possible
to pump the helium down to superfluid temperatures and enables direct
observation of the liquid.

\subsubsection{Transition between He I and He II}

The most obvious phenomenon is the transition between helium I and helium II. As the temperature is lowered through the $\lambda$-point, it is possible to observe through the window the marked change in the appearance of the liquid, which is a result of helium II's inability to support temperature gradients as described in section \ref{thermalconductivity}.

\subsubsection{Superfluid fountain}\label{superfluidfountain}

The next experiment allows for a particularly impressive display of this inability to support temperature gradients. A specially crafted glass container has a very narrow capillary tube on top, and a wide semi-porous glass bottom. The semi-porous glass (made of very finely ground and compacted glass) creates winding openings of diameter on the order of $100$ $\mu$m which allow only superfluid, not normal fluid, to pass. A $100$ $\Omega$ resistor placed inside the container acts as a heater. When a potential difference is applied to the resistor, the heated helium draws cooler superfluid to it in an effort to remove the temperature gradient; this sudden rush of superfluid forces some fluid out through the capillary tube on top, creating a small helium fountain.

\subsection{Second Sound Experiment}

In the second part of the experiment, we measure the propagation speed of second sound as a function of temperature.

Second sound is the wave propagation of heat in a superfluid, so a source of heat is necessary. A grid of Nichrome wire, with a resistance of about $10$ $\Omega$, is used as a resistive heater. A HP3018A function generator creates an $80$ $\mu$s low voltage ($5$ V) pulse, which is amplified to $40$ V with a Kepco BOP 50-2M Power Amplifier and sent to the heater. 

To measure second sound propagation speed, a thermometer is necessary
to detect the signal. A carbon resistor ($65$ $\Omega$ at room
temperature) is supplied with a constant $100$ $\mu$A current, and the
voltage across it is read by a voltmeter. A change in temperature of
the resistor at very low temperatures corresponds to a change in
resistance, and therefore a change in voltage. This resistor (referred
to as a bolometer) is placed at the end of a sliding rod, whose
distance from the Nichrome wire is able to be adjusted from $3$ cm to
$12$ cm.

A heat pulse read on the bolometer is expected to be visible as a dip
on an oscilloscope. As temperature increases when the heat pulse
passes the bolometer, the resistance decreases, and therefore the
voltage decreases. The signal from the bolometer is very noisy and
difficult to read, as the wires leading to it are unshielded so as to
lower thermal contact into the liquid helium. A $60$ Hz bandpass
filter reduces powerline pickup, the worst source of noise; to remove
other sources of noise and to amplify the signal, a EG\&G PARC 113
Pre-Amp is set to a gain of $10^4$ with a $3\times 10^3$ Hz high-pass
filter and $0.1$ Hz low-pass filter.

To perform each measurement at a constant temperature, it is important to maintain the vapor pressure at a constant level. A series of valves attached to the vacuum pump allows for very fine control of the pumping rate, which can be set at a level to temporarily balance the heat flow from the outside environment. Changes in volume of the liquid offset this delicate equilibrium, so it is necessary to constantly monitor the pressure and update the pumping rate to compensate.

In order to calculate the second sound velocity, first a constant temperature is selected. A series of heat pulses are sent using the function generator, and an oscilloscope is set to average over $128$ measurements of the response from the bolometer to further increase signal to noise. The distance between the wire and the bolometer is recorded as well as the time between the pulse from the generator and the first dip on the bolometer; this time corresponds to how long it took the pulse to travel from the nichrome wire to the bolometer. The bolometer distance is then altered, and the measurement repeated. Once $7$ measurements are taken at one temperature, the temperature is changed and the process is repeated.

\subsection{Heat Capacity Experiment}\label{heatcapacityexperiment}

The purpose of the next experiment is to determine the specific heat
of liquid helium as a function of temperature; to measure this
dependence, a predetermined heat pulse is sent into a high purity sample
of liquid helium contained in a small copper cell and the temperature
response is measured at different initial temperatures.

\subsubsection{Thermometer Calibration}\label{thermometercalibration}

As in the second sound experiment, a resistor attached to the cell is
used to monitor changes in temperature. Unlike the second sound
experiment, the resistor used is composed of germanium, which has
predictable behavior as a function of temperature and excellent
cycling properties (that is, one measurement of a resistance will
always correspond to the same temperature, as there is little drift
between experimental runs). A constant current of $1.00$ $\mu$A is
applied to the thermometry resistor by tuning the voltage of the power
supply and reading the electric potential drop across a fixed $9.963$
k$\Omega$ resistor at room temperature which is in series with the
thermometer. The potential difference across the thermometry resistor
is then amplified by an operational amplifier circuit with a gain of
170 so as to increase the signal and decrease output impedance
\footnote{Note that the maximum current is limited by the maximum
  output voltage of the op-amp, 4.4 Volts. While it would be ideal to
  use a higher current in order to increase resolution of the
  potential difference measurement, since the error is $\delta V
  \propto \frac{V}{I} \delta I$ and therefore goes down with increased
  current, the op-amp enforces a maximum of 1 $\mu$A to avoid
  saturation.}. 

A large volume of exchange gas is introduced into the the inner vacuum
can in order to couple the temperature of the cell (and thereby the
cold resistor) with the temperature of the helium bath. Such a large
amount is necessary because of the activated charcoal cryopumps
placed in the vacuum can: these must be saturated to provide proper
coupling. As the temperature decreases, the efficiency of the cryopump
increases, so it is important to monitor the pressure of this exchange
gas and ensure that it is always present. The potential difference
across the germanium resistor and the vapor pressure of the helium
bath are recorded as the helium is pumped down to equilibrium. Pumping
is paused as each potential difference measurement is taken, allowing
the systems to thermally equilibrate. After adjusting values of the
recorded pressure for known offsets in the gauges, the potential
difference is fit against temperature with the function

\begin{equation}
\label{eq:fiteqn}
\ln{V} = \sum_{n=0}^{8} a_{n} (\ln{T})^{n}
\end{equation}

recommended by White for fitting the resistance of germanium as a
function of temperature \cite{white}. Note that we fit voltage, which
is only different from resistance by a factor of $I$, changing the
constants but not the overall form of Eqn.\ref{eq:fiteqn}.

\subsubsection{Heat Pulses}

Once the germanium resistor successfully calibrated, the next step is
to send constant heat pulses into the system. A second resistor is
placed onto the copper cell to function as a heating element. A four
probe measurement of the resistance is taken; for a potential
difference pulse of a set duration and amplitude, the output heat is given by

\begin{equation}
Q = P t = V^{2} \Delta t /R
\end{equation}
where $V$ is the magnitude of the pulse, $\Delta t$ is the duration,
and $R$ is the resistance. Note that the electric potential here is
read off from the 4-probe measurement as well, not from the voltage
source itself, in order to cancel the resistance of the long wires
down to the resistor. An electronics system sends out pulses with
adjustable amplitude and duration, so the rate of pulses and amount of
heat per pulse can be precisely regulated.


\subsubsection{Heat Capacity of Copper Cell}

Before the specific heat of helium can be determined, it is necessary
to measure the heat capacity of the empty copper cell. To begin, we
pump down the system to the lowest temperature possible. At this point
the inner vacuum is still empty, so the cell is still at room
temperature. A moderate amount of helium buffer gas is introduced into
the vacuum can, which the cryopump quickly absorbs. A heater attached
to the cryopump is then activated (and the temperature monitored on a
diode thermometer) so as to desorb the cryopump and allow the copper
cell to equilibrate with the helium bath at the very low
temperature. The heater is then turned off and the buffer gas absorbed
by the cryopump, so the copper cell is isolated from the bath. Because
the capillary leading out is very small, coupling to the outside
environment is minimal, though not completely negligible. The potential
difference across the germanium resistor is recorded by a computer,
and pulses of heat are sent into the cell. Sufficient time is allowed
between pulses for the system to equilibriate. With a given heat pulse
and a measured temperature response per pulse, the heat capacity of
the copper cell (as a function of temperature) can be calculated and
fit to a curve, as described in \ref{specificheatofmetals}.

\subsubsection{Specific Heat of Helium}
In order to measure the specific heat per mole of helium, the number
of moles in the high purity sample must be established. The helium is
stored in a container of known volume at room temperature; once the
pressure is measured, the ideal gas law $P V = n R T$ gives the number
of moles $n$. Next, the system is cooled to the lowest achievable
temperature and the cell is brought into thermal contact with the bath
using the procedure described above. The helium from the container is
then allowed into the capillary connecting it to the cell; because
liquid helium is $760$ times as dense as gaseous helium at a given
pressure \cite{shi}, almost all of the helium condenses into the cell,
and the only gas that remains is from vapor pressure of the liquid at
this low temperature. 

The cryopump heater is then turned off so as to isolate the system,
and predetermined heat pulses are sent into the cell through the
heater. \footnote{We make the approximation that all the helium
  becomes liquid, and that this is true for all the relevant
  temperatures. This assumption introduces a minor error, as the vapor
  pressure does change and the amount of gaseous helium therefore also
  changes, but this is a negligible effect compared to the factor of
  $760$ discussed previously.} The potential difference across the
thermometer is once again read and recorded by a computer. It is
particularly important at this step to ensure that enough time has
been provided for the system to equilibrate after each pulse; while
helium II does not support temperature gradients and therefore
equilibrates quickly because the copper is the relevant time constant,
helium I takes much longer to reach steady temperature. Given that
there is an element of heating through the capillary which couples the
system to room temperature, it is also important ensure the time
between pulses is short enough such that the heating does not affect
measurements.  Once the heat capacity of the entire system is
calculated, the heat capacity of the copper is subtracted and the
result divided by the number of moles, so as to obtain the molar
specific heat of the helium alone.
