%Experimental body
%Created MB 04-12

\section{Experimental}\label{experimental}

Several features are common to all three experiments performed. Two separate dewars are used in the experiment; both are Precision Cryogenic Systems model PVS-337 LHe Vapo-Shield Dewars. Aluminized Mylar superinsulation provides isolation to the external environment. The preparatory procedure for all experiments is to precool the chosen dewar to 77 K using liquid nitrogen and then to fill the dewar with liquid helium. Temperature control of the liquid helium is achieved by pumping on the liquid with a vacuum pump: changing the vapor pressure also changes the temperature in a predictable manner, subject to a calibration found in (something).

%note uniformity of temperature when pumping down

\subsection{Visual Observation Experiments}

The first experiment consisted of visual observations of different phenomena in superfluid helium. One helium dewar has a glass observation window near the bottom; while this increases coupling to the environment and thereby raises the lowest achievable temperature, it is still possible to pump the helium down to superfluid temperatures.

\subsubsection{Transition between He I and He II}

The most obvious phenomena is the transition between helium I and helium II. As the temperature is lowered through the lambda-point, it is possible to observe through the window the marked change in visual character of the liquid, which is a result of helium II's inability to support heat gradients and boiling.

\subsubsection{Superfluid fountain}

The next experiment allows for a particularly impressive method of displaying this inability to support heat gradients. A specially crafted glass container has a very narrow capillary tube on its top, and a wide semi-porous glass bottom. The semi-porous glass (made of very finely ground and compacted glass) allows only superfluid, not normal fluid, to pass. A 100 $\Omega$ resistor placed inside the container acts as a heater. When a voltage is applied to the resistor, the heated superfluid draws cooler superfluid to it in an effort to remove the temperature gradient; this sudden rush of superfluid forces some out through the capillary tube on top, creating a small helium fountain.

\subsection{Second Sound Experiment}

In the next experiment, we measure the propagation speed of second sound as a function of temperature.

Second sound is the propagation of heat in a superfluid as a wave, so a source of heat is necessary. A grid of Nichrome wire, with a resistance of about 10 $\Omega$ is used as a resistive heater. A HP3018A function generator creates an 80 $\mu$s low voltage (5 V) pulse, which is amplified to 40 V with a Kepco BOP 50-2M Power Amplifier and sent to the heater. 

To measure the actual propagation speed, a thermometer is also necessary. A carbon resistor (nominally 65 $\Omega$ at room temperature) is supplied with a constant 100 $\mu$A current, and the voltage across it is read by a voltmeter. A change in temperature of the resistor at very low temperatures corresponds to a change in resistance, and therefore a change in voltage. This resistor (referred to as a bolometer) is placed at the end of an adjustable sliding rod, whose distance from the Nichrome wire is able to be adjusted.

A heat pulse read on the bolometer is expected to be visible as a dip on an oscilloscope. As the temperature goes up from the pulse, the resistance goes down, and therefore the voltage goes down. The signal from the bolometer is very noisy and difficult to read, as the wires leading to it are unshielded so as to lower thermal contact into the liquid helium. A 60 Hz bandpass filter removes the worst source of noise; to remove other sources of noise and to amplify the signal, a EG\&G PARC 113 Pre-Amp is set to a gain of 10,000 with a 3 kHz high-pass filter and .1 Hz low-pass filter.

To perform measurements at a constant temperature, it is important to maintain the vapor pressure at a constant level. A series of valves attached to the vacuum pump allows for very fine control of the pumping rate, which can, with some persistence, be set at a level to temporarily balance out the heating from the outside environment. Changes in the volume of the liquid offset this delicate equilibrium, so it is necessary to constantly monitor the pressure and update the pumping rate to compensate.

To perform measurements of the velocity of second sound, first a constant temperature is selected. A series of heat pulses is sent using the function generator, and an oscilloscope averages over the response from the bolometer to further reduce noise. The time between the pulse from the generator and the first dip on the bolometer is recorded; this time corresponds to how long it took the pulse to travel from the nichrome wire to the bolometer. The distance between the wire and the bolometer is then altered and recorded, and the measurement repeated. Once 7 measurements are taken at one temperature, the temperature is changed and the process is repeated.

\subsection{Heat Capacity Experiment}\label{heatcapacityexperiment}

The purpose of the next experiment is to determine the specific heat of liquid helium as a function of temperature; to measure this, a pulse of known heat is sent into a high purity sample of liquid helium in a small cell and the temperature response is measured at different starting temperatures.

The first step of monitoring the temperature response is to create a thermometer; as in the second sound experiment, a resistor attached to the cell is used to monitor changes in temperature. Unlike in the second sound experiment, the resistor used is composed of germanium, which has very predictable behavior as a function of temperature (resistance goes up with decreasing temperature) and excellent cycling properties (that is, one measurement of a resistance will always correspond to the same temperature, as there is little drift between experimental runs). A constant current of 1.00 $\mu$A is able to applied to the cold resistor by tuning the voltage of the power supply and reading the voltage drop across a 9.963 k$\Omega$ resistor at room temperature which is in series with the thermometer. The voltage across the cold resistor is then amplified by an op-amp circuit with a gain of 170 so as to increase the signal and decrease output impedance. \footnote{Note that the maximum current is limited by the maximum output voltage of the op-amp, 4.4 Volts. While it would be ideal to use a higher current in order to increase resolution of the voltage measurement, since the error of the voltage measurement is $\delta V \propto \frac{V}{I} \delta I$ and therefore goes down with increased current, the op-amp enforces a maximum of 1 $\mu$A to avoid saturation.} A large volume of exchange gas is introduced into the the inner vacuum can in order to couple the temperature of the cell (and thereby the cold resistor) with the temperature of the helium bath. Such a large amount is necessary because there of the activated charcoal cryopumps placed in the vacuum can: these must be saturated to provide proper coupling As the temperature decreases, it is important to monitor the pressure of this exchange gas and ensure that a high amount is always present, as the efficiency of the cryopump increases at low temperatures. The voltage across the germanium resistor and the vapor pressure of the helium bath are recorded as the helium is pumped down to equilibrium at approximately (SOMETHING K). Pumping is paused as each voltage measurement is taken, allowing the systems to thermally equilibrate. After adjusting values of the recorded pressure for known offsets in the gauges, the voltage is fit against temperature with the function

\begin{equation}
\label{eq:fiteqn}
\ln{V} = \sum_{n=0}^{8} a_{n} (\ln{T})^{n}
\end{equation}

recommended by White for fitting the resistance of germanium as a function of temperature (we fit voltage, which is only different from resistance by a factor of $I$) \cite[p.~140]{white}.

With the germanium resistor successfully calibrated, the next step is devise a method for sending pulses of heat into the system. A second resistor is placed onto the copper cell to function as a heating element. A four probe measurement of the resistance is taken; for a voltage pulse of a set length, we now have $Q = P t = V^{2} t / R$. Note that the voltage here is read off from the 4-probe measurement as well, not from the pulser itself: the resistance of the wire down to the resistor does dissipate some voltage. A small electronics box sends out pulses of a chosen voltage and duration at the press of a button, so the rate and exact amount of heat per pulse can be chosen exactly.

Before the specific heat of helium can be determined, it is necessary to measure to heat capacity of the copper cell by itself. To begin, we pump down the system to the lowest temperature possible. At this point the inner vacuum is still empty, so the inner cell is still at room temperature. A moderate amount of helium buffer gas is introduced into the vacuum can, which the cryopump quickly absorbs. A heater attached to the cryopump is then activated (and the temperature monitored on a diode thermometer next to it) so as to desorb the cryopump and allow the copper cell to equilibrate with the helium bath at the very low temperature. The heater is then turned off and the buffer gas absorbed by the cryopump, so the copper cell is isolated from the bath (and because the capillary leading out is very small, coupling to the outside environment is minimal as well). Voltage across the germanium resistor is recorded by a computer, and pulses of heat are sent into the cell (making sure that there is enough time between pulses for the heat to distribute evenly through the cell). With a given heat pulse and a measured temperature response per pulse, the heat capacity of the copper cell (as a function of temperature) can easily be calculated and fit to a curve, as described in (SOMETHING).

With the heat capacity of the copper cell calculated, it is finally possible to measure the specific heat of helium. First, the number of moles in the high purity sample must be calculated. The helium is stored in a bottle of known volume at room temperature; once the pressure is measured, the ideal gas law $P V = n R T$ allows for the calculation of the number of moles. Next, the system is cooled to the lowest temperature again and the cell is brought into thermal contact with the bath with the procedure described above. The helium from the bottle is then allowed into the capillary connecting it to the cell; because liquid helium is 700 times as dense as gaseous helium at a given pressure (CITE THIS), almost all of the helium condenses into the cell, and the only gas that remains is from vapor pressure of the liquid at this lowest temperature. The cryopump heater is then turned off so as to isolate the system once more, and known heat pulses are sent into the cell through the heater. \footnote{We make the approximation that all the helium becomes liquid, and that this is true for all the relevant temperatures. Of course there is an element of inaccuracy to this, as the vapor pressure does change and the amount of gaseous helium therefore also changes, but this is a small effect compared to the factor of 700 discussed previously.} The voltage is once again read and recorded by a computer. It is particularly important at this step to ensure that enough time has been provided for the system to equilibrate after each pulse; while superfluid helium does not support temperature gradients and therefore equilibrates quickly (the copper part and normal part of the fluid is the relevant time constant here), helium I takes much longer to settle. Given that there is also an element of heating through the capillary, the top end of which is at room temperature, it is also important to not wait too long, so that heating does not dominate.  With these measurements made, the heat capacity of the entire system is calculated. The heat capacity of the copper is then subtracted off, and the result divided by the number of moles calculated earlier, so as to obtain the specific heat of the helium alone.