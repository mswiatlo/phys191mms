%nonlinear regression error body
%ss

\section{Nonlinear Regression Analysis}\label{nonlinearregressionanalysis}
In this experiment, most of the curve fitting is done using $n$-parameter nonlinear regression where the data is approximated by

\begin{equation}
\label{nonlinfit}
y_{i}=f(\mathbf{\beta},\mathbf{x_{i}'})+\epsilon_{i}
\end{equation}

where $y_{i}$ is the $i$th expected value, $\mathbf{\beta}$ is a parameter vector, $\mathbf{x_{i}'}$ is the $i$th row of predictors, and $\epsilon_{i}$ is the associated random error \cite{bates}.  The likelihood, $L$, is given by 

\begin{equation}
\label{likelihood}
L(\beta,\sigma^{2})=\frac{1}{(2\pi\sigma^{2})^{n/2}}\exp(-\frac{\sum_{i=1}^{n}[y_{i}-f(\beta,\mathbf{x_{i}'})]^{2}}{2\sigma^{2}})
\end{equation}

where $n$ is the number of data points and $\sigma^{2}$ is the variance \cite{bates}. $L$ is maximized where the sum of squared errors, $S(\beta)$, given by \cite{bates}

\begin{equation}
\label{sse}
S(\beta) = \sum_{i=1}^{n}[y_{i}-f(\mathbf{\beta},\mathbf{x_{i}'})],
\end{equation}

is minimized.