%Results and Discussion body
%Created MS 1-30

\section{Results and Discussion}

\subsection{Determination of Muon Lifetime}

By constructing the logic (system) discussed in the previous section, we can determine when a muon comes to rest in the middle scintillator and correspondingly produce a signal denoted the START signal.  Likewise, we can determine when the stopped muon decays by observing the production of an electron with downward velocity.  The signal that is produced is denoted the STOP signal.  These signals can then be passed from a scope to a LabView program via GPIB.  By observing the duration between START and STOP pulses the time we are able to measure the time it takes for a stopped muon to decay.  Essentially this time is the lifetime of an individual muon.  

By measuring and recording these durations 

\subsubsection{Error Analysis}

\subsection{Determination of Muon Mass}

\subsubsection{Error Analysis}

\subsection{Determination of Weak Force Coupling Constant}

