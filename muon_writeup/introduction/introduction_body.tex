%Introduction body
%Created MS 1-30

\section{Introduction}\label{introduction}

The muon is a fundamental particle produced in the upper atmosphere as
a secondary product of cosmic ray collisions with atmospheric
molecules. It decays via the weak interaction into an electron and two
neutrinos, with a mean decay lifetime of $2.2~\mu$s, longer than
every known particle other than the neutron \cite{pdg}. With muons
comprising most of the cosmic ray flux at sea level, the muon is a
good candidate for the study of the weak force \cite[p.~8]{rossi}.

Our experiment consists of two main components: the muon lifetime
measurement and the muon mass measurement. In
Section~\ref{background}, \emph{Background}, we introduce the
theoretical basis for muon creation and decay as well as the
motivation for lifetime and mass measurements and $G_{F}$ calculations. We describe the
experimental setup for muon detection which consists of a system of
three scintillators and photomultiplier tubes (PMTs) in
\emph{Instrumentation} (Section~\ref{instrumentation}). Using this
system, the cosmic ray muons passing through the scintillators and
their decay products can be detected, along with the energy of these
particles (as described in Section~\ref{procedure}, \emph{Procedure}). In Section~\ref{resultsanddiscussion}, \emph{Results and Discussion}, the muon lifetime and
mass results are presented with the relevant statistical analysis of
data and compared to previous experimentally established
values. Finally, we use the muon mass and lifetime values to calculate
the Fermi coupling constant $G_F$ that describes the strength of weak
interactions.
