%Introduction body
%Created MS 1-30

\section{Introduction}

The muon, a fundamental particle produced in the upper atmosphere as a
secondary product of cosmic ray collisions, was originally discovered
in 1936 \cite{}. It decays via the weak interaction with a mean decay
lifetime of $2.2 \mu s$, longer than every known particle other than
the neutron \cite{}. With muons comprising $80$\% of cosmic ray flux at
sea level, the muon is a good candidate for the study of the weak
force \cite{}.

Our experiment consists of two main components: the muon lifetime
measurement and the muon mass measurement. We describe the
experimental setup which consist of a system of three scintillators
and Photomultiplier Tubes (PMTs). Using this system, the cosmic ray
muons and their decay products are detected along with their
enegy. The muon lifetime and mass results are presented with the
relevant statistical analysis of data, and compared to previous
experimentally established values. Finally, the muon mass and lifetime
values are used to calculate the weak force coupling constant,
$G_F$.
