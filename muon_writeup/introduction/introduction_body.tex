%Introduction body
%Created MS 1-30

\section{Introduction}\label{introduction}

The muon is a fundamental particle produced in the upper atmosphere as
a secondary product of cosmic ray collisions. It decays via the weak
interaction with a mean decay lifetime of $2.2 \mu s$, longer than
every known particle other than the neutron \cite{pdg}. With muons
comprising most of the cosmic ray flux at sea level, the muon is a
good candidate for the study of the weak force \cite[p.~8]{rossi}.

Our experiment consists of two main components: the muon lifetime
measurement and the muon mass measurement. In section
~\ref{background}, \emph{Background}, we introduce the theoretical
basis for these measurement as well as that of muon creation and
decay. We describe the experimental setup which consists of a system
of three scintillators and photomultiplier tubes (PMTs) in
\emph{Instrumentation}~\ref{instrumentation}. Using this system, the
cosmic ray muons passing through the scintillators and their decay
products can be detected along with their energy (\emph{Procedure},
~\ref{procedure}). The muon lifetime and mass results are presented in
\emph{Results and Discussion} ~\ref{resultsanddiscussion} with the
relevant statistical analysis of data, and compared to previous
experimentally established values. Finally, we use the muon mass and
lifetime values to calculate the Fermi coupling constant $G_F$ which
describes the strength of weak interactions.
