%Background body
%Created MB 03-01

\section{Background}\label{background}

The muon is a lepton, originally discovered by Carl Anderson in 1936
\cite{anderson}. Muons ($\mu^-$) and antimuons ($\mu^+$) are the most
numerous charged particles at sea level\cite{pdg}. Most of them are
decay products of pions and kaons, mesons produced at a height of
about $15$ km via the interaction of cosmic ray particles with the
Earth's upper atmosphere \cite{amsler}. The muon is produced in weak
decays as shown in Figure~\ref{figure:pidecay}, and at sea
level forms $80\%$ of the total cosmic ray flux \cite[p.~8]{rossi}. This very 
high flux rate makes the muon an ideal particle for studying weak interactions.

\begin{figure}[h]
\begin{center}
\fcolorbox{white}{white}{
  \begin{picture}(352,118) (131,-75)
%    \caption{blahblahblah}
    \SetWidth{1.0} \SetColor{Black}
    \Line[arrow,arrowpos=0.5,arrowlength=5,arrowwidth=2,arrowinset=0.2](160,22)(256,-26)
    \Line[arrow,arrowpos=0.5,arrowlength=5,arrowwidth=2,arrowinset=0.2](256,-26)(160,-74)
    \Line[dash,dashsize=10,arrow,arrowpos=0.5,arrowlength=5,arrowwidth=2,arrowinset=0.2](256,-26)(352,-26)
    \Line[arrow,arrowpos=0.5,arrowlength=5,arrowwidth=2,arrowinset=0.2](352,-26)(432,22)
    \Line[arrow,arrowpos=0.5,arrowlength=5,arrowwidth=2,arrowinset=0.2](432,-74)(352,-26)
    \Text(144,22)[lb]{\large{\Black{$d$}}}
    \Text(144,-74)[lb]{\large{\Black{$u$}}}
    \Text(128,-26)[lb]{\large{\Black{$\pi^{-}$}}}
    \Text(448,22)[lb]{\large{\Black{$\mu^{-}$}}}
    \Text(448,-74)[lb]{\large{\Black{$\nu_{\mu}$}}}
    \Text(304,-10)[lb]{\large{\Black{$W^{-}$}}}
  \end{picture}
}
\caption{The weak decay of $\pi^-$ which produces $\mu^-$: $\pi^- \rightarrow
\mu^-~\nu_{\mu}$}
\label{figure:pidecay}
\end{center}
\end{figure}


\subsection{Muon Decay}

In free space, negatively charged muons decay weakly into an electron,
muon neutrino, and electron antineutrino \cite{easwar}
(Figure~\ref{figure:mudecay}):
\begin{equation}\mu^- \rightarrow e^-~\nu_{\mu}~\overline{\nu_e}\label{mudecay}\end{equation}

with a corresponding antimatter process:
\begin{equation}\mu^+ \rightarrow e^+~\overline{\nu_{\mu}}~{\nu_e}\label{antimudecay}\end{equation}

\begin{figure}[h]
\begin{center}
\fcolorbox{white}{white}{
  \begin{picture}(336,166) (131,-75)
    \SetWidth{1.0}
    \SetColor{Black}
    \Line[arrow,arrowpos=0.5,arrowlength=5,arrowwidth=2,arrowinset=0.2](160,6)(304,6)
    \Line[arrow,arrowpos=0.5,arrowlength=5,arrowwidth=2,arrowinset=0.2](304,6)(416,70)
    \Line[dash,dashsize=10,arrow,arrowpos=0.5,arrowlength=5,arrowwidth=2,arrowinset=0.2](304,6)(368,-42)
    \Line[arrow,arrowpos=0.5,arrowlength=5,arrowwidth=2,arrowinset=0.2](368,-42)(416,-10)
    \Line[arrow,arrowpos=0.5,arrowlength=5,arrowwidth=2,arrowinset=0.2](416,-74)(368,-42)
    \Text(128,6)[lb]{\large{\Black{$\mu^{-}$}}}
    \Text(432,-10)[lb]{\large{\Black{$e^{-}$}}}
    \Text(432,-74)[lb]{\large{\Black{$\nu_{e}$}}}
    \Text(432,70)[lb]{\large{\Black{$\nu_{\mu}$}}}
    \Text(304,-42)[lb]{\large{\Black{$W^{-}$}}}
  \end{picture}
}
\caption{Weak decay of $\mu^-$.}
\label{figure:mudecay}
\end{center}
\end{figure}

The decay of the muon is described by the exponential function:

\begin{equation} N(t) = N_0 e^{-\Gamma_{\mu} t} \label{expdecay}\end{equation} 

where $\Gamma_{\mu}$ is the decay rate. In our first experiment 
(as described in Section~\ref{muonlifetimemeasurement}), we seek to
measure the characteristic lifetime of the decay, $\tau_{\mu} =
1/\Gamma_{\mu}$.

%we measure the time between the start event, when
%a $\mu^- (\mu^+)$ comes to rest in a scintillator in the lab and the
%stop event, which signals the emission of $e^- (e^+)$ in the muon
%decay. The histogram of the recorded times is then fit to
%\eqref{expdecay} to give the lifetime of the muon.

\subsubsection{Decay in Matter}\label{decayinmatter}

In matter, another decay channel is possible for $\mu^-$ via nucleus
capture:

\begin{equation}\mu^-~p \rightarrow n~\nu_{\mu} \label{pcap} \end{equation}

Due to the relatively faster process of decay via proton capture, the
mean lifetime of $\mu^-$ in matter is shortened relative to that in
free space and depends on the material. Since the positive $\mu^+$ is
repelled by the nucleus, the $\mu^+$ lifetime is unaltered in matter.

The likelyhood of the $\mu^-$ capture is proportional to $Z^4$, where
$Z$ is the atomic number of the material; for light elements, the
effect of this process is minimal \cite[p.172]{rossi}. For carbon, for
instance, $Z=6$ and the mean lifetime of the muon is theoretically
predicted to be between $1.5$ and $1.9\mu$s \cite[p.~170]{rossi}.

In addition, the products of the decay desribed in Equation
~\eqref{pcap} are a neutron and a neutrino. Because neutrons carry no
charge, the
efficiency of the scintillator in detecting neutrons is lower than that of
electrons and positrons from the decays in Equations
\eqref{mudecay} and \eqref{antimudecay}.  Thus, we expect the effect of muon capture on the measured
lifetime to be small, although it may slightly decrease the value from
the experimentally established $2.197~\mu$s in free space \cite{pdg}.

\subsection{Effects of Relativistic Time Dilation}

Even with velocities within a percent of the speed of light, the
travel time of the muon from the point of creation in the atmosphere
takes approximately $50\mu$s - over 20 decay lifetimes - to reach the
ground for a muon emitted in the downward direction. According to
Newtonian physics, the flux would be reduced by a factor of over
$10^{10}$ over the time of flight, and muons would be undetectable at
sea level. However, the flux of muons at sea level, where the lab is
located, remains large at $10^{-2}$ cm$^{-2}$s$^{-1}$sr$^{-1}$:
reduced by a factor of just $5$ from the peak flux at $15$~km
\cite{rossi}.

This effect is due to time dilation predicted by the theory of special
relativity. While in the frame of the laboratory the time of flight of
the muons is $50\mu$s, the muon itself experiences a proper time
reduced by a factor of $\gamma$: $ \gamma t_{\mu} = t_{lab}$, where
$\gamma = 1/\sqrt{1 - \frac{v_{\mu}^2}{c^2}}$. Since the particles are
travelling close to the speed of light, the relativistic correction
becomes non-negligible. With muon velocities ranging from $.994c$ to
$.998c$, the proper time experienced by the muon is between $3.2$ and
$5.5~\mu$s, less than $2$ lifetimes on average, constistent with
detected muon flux values.

The time in flight is still on the same order as, and even greater
than, the lifetime of muon decay that our experiment seeks to
measure. Nevertheless, the time the muons experience in the atmosphere
prior to stopping in the detector has no effect on the decay rate
measurement. While we do not sample the quickest decaying electrons,
this amounts to simply cutting off the lowest end of the exponential: the actual
decay parameter $\Gamma_{\mu}$ is not changed by eliminating this data.

%While we do sample fewer short decay times and detect
%fewer slow moving muons, this fact simply decreases the amount of data
%without affecting the parameters of the exponential.

\subsection{Muon Mass}\label{muonmass}

Another experimental setup (Section~\ref{muonmassmeasurement}) can be
used to make a measurement of the muon mass.

In order to measure muon mass, we consider the products of $\mu^-$
decay: an electron and two neutrinos (Equation~\eqref{mudecay}); the
antimatter decay analysis is identical. For a muon which stops and
decays in the scintillator, the center of mass frame is the same as
the lab frame. To a good approximation we can assume that the rest
energy of the muon is fully converted to the kinetic energy of the
$e^-$ and neutrinos, as the electron mass is only $0.5\%$ of the muon
mass and the neutrinos are essentially massless. Then, measuring the
energy distribution of the emitted electrons will provide information
regarding the initial muon mass. Specifically, due to conservation of
momentum, the magnitude of electron momentum, $p_e$, must equal the
sum of the neutrino momenta. We can see this from the following
argument. Fixing $p_e$ along the x-axis, the only possible scenario of
the decay is pictured in Figure~\ref{figure:e_nu}, where $0 \leq
\theta <\pi/2$ and is measured from the negative x-axis.

\begin{figure}[ht]
\begin{center}
\input{./figures/electron_neutrino.pstex_t}
\caption{\small{Decay products of the muon. Due to conservation of
energy and momentum, momenta of the two neutrinos must be at equal
angles, $\theta$ and $-\theta$, from the direction of electron
momentum. The configuration with $\theta = 0$ maximizes the electron
momentum (and therefore energy).}}
\label{figure:e_nu}
\end{center}
\end{figure}

The total momentum of the electron is maximized when the neutrino
momenta have no $y$ component, that is $\theta = 0$ (an identical argument
eliminates any possible $z$ component as well). In this case,
$p_e = p_e^{max} = \Sigma p_{\nu}$ by conservation of momentum. Again
neglecting electron mass, we have $E_e = p_ec$, so the energy of the
electron is maximized when its momentum is maximized. Thus, the
maximum kinetic energy of the electron is half the rest energy of the
muon, as claimed.

\begin{equation}E_e^{max} = \frac{1}{2}E_{\mu} = \frac{1}{2}m_{\mu}c^2\end{equation}

By measuring the energy spectrum of emitted electrons, which is a
$\beta$ decay spectrum with a cutoff at $\frac{1}{2}m_{\mu}c^2$, we
can find the maximum electron energy and thereby determine the muon
mass.

\subsubsection{Energy Calibration}\label{energycalibration}

Our instruments enable us to measure the maximum voltage of a pulse
from an electron that registers on the PMT (the method is described in
more detail in Section~\ref{muonmassmeasurement}). In order to convert
the pulse height distribution to an energy distribution, we have to
calibrate the pulse height voltage in terms of energy lost by the
particle in the detector.

To do so, we find the muon pulse height voltage which corresponds to
the minimum stopping power of the muon, as a function of incoming momentum ($\langle -\frac{dE}{dx}\rangle$, approximately $2$ MeV
g$^{-1}$cm$^{2}$). This voltage is given by the mode in the peak
height distribution of muons which pass through all three
scintillators. Since the through going muons have a relatively random
distribution of momenta, the most frequent rate of energy loss will be
near a local extremum in the stopping power vs. momentum function,
which in this case is a minimum (see Appendix~\ref{masscalibration}
for more details).

Assuming the scintillator light output varies linearly with the amount
of energy deposited, and measuring the scintillator density and
thickness, we find a conversion ratio between pulse voltage and
electron energy.

\subsection{Weak Force Coupling Constant}

The decay rate of the muon $\Gamma_{\mu}$ is proportional to the
square of the amplitude of the decay diagram (Figure
~\ref{figure:mudecay}), which depends on the product of the couplings
at each vertex. In this case, the coupling at each of the two vertices
is proportional to $\sqrt{G_F}$, the Fermi coupling constant, so we
have

\begin{equation}\Gamma_{\mu} \propto G_F^2 \end{equation}

A more involved calculation \cite[p.~310-314]{griffiths} gives that the
lifetime of the muon is

\begin{equation}\tau_{\mu} = \dfrac{192\pi^3\hbar^7}{G_F^2m_{\mu}^5c^4}\label{gf}\end{equation}
where $c$ is the speed of light, $\hbar$ is Planck's constant, and
$m_{\mu}$ is the rest mass of the muon. 

Once we establish the values of $\tau_{\mu}$ and $m_{\mu}$, we can
find the Fermi coupling constant $G_F$, which describes the strength
of the weak interaction\footnote{Although $G_F$ is not equivalent to
the weak coupling constant, $g_w$, they are related by the equation
\[G_F\equiv \frac{\sqrt{2}}{8}\left(\frac{g_w}{M_Wc^2}\right)^2(\hbar c)^3\]
where $M_W$ is the mass of the $W$ bosons which mediate the weak
interaction. Thus $G_F$ is sufficient and is commonly used to describe
weak interaction formulas\cite[p.~313]{griffiths}}.

The weak decay of the muon is the clearest of all weak interaction
phenomena in both its experimental and theoretical aspects. Further
considering the easy availability of muons at sea level, the
muon decay is an effective means of studying the nature of the weak
force, and specifically finding the Fermi coupling constant $G_F$.
