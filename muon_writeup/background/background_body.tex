%Background body
%Created MB 03-01
%\usepackage{graphicx}
%\usepackage{setspace}
%\usepackage{color}
%\usepackage{amssymb}
%\usepackage{listings}
%\usepackage{epstopdf}

\section{Background}

\subsection{}
Muons ($\mu^-$) and antimuons($\mu^+$) are the most numerous charged
particles at sea level\cite{pdg}, most of them produced at a height of
about $15$ km from decays of charged kaons and pions that are formed
from the interaction of cosmic ray particles with the Earth's upper
atmosphere \cite{amsler}. The muon is produced in a weak decays as
shown for example in Figure and at sea level forms $80\%$ of cosmic
ray flux.


In free space, negatively charged muons decay weakly into an
electron, muon neutrino, and electron antineutrino \cite{easwar}
(Figure):

\begin{equation}\mu^- \rightarrow e^- \nu_{\mu}\overline{\nu_e}\label{mudecay}\end{equation}

with a corresponding antimatter process of

\begin{equation}\mu^+ \rightarrow e^+ \overline{\nu_{\mu}}{\nu_e}\label{antimudecay}\end{equation}

The muon lifetime is approximately $2.2 \mu$s\cite{easwar}, second
only to the lifetime of the neutron. In matter, another decay is
possible for $\mu^-$ via nucleus capture:

\begin{equation}\mu^-p^+ \rightarrow n \nu_{\mu} \label{pcap} \end{equation}

The lifetime asssociated with nuclear capture is typically shorter
than the free muon lifetime. However, the products released in process
\eqref{pcap} are neutral, and therefore not detected in our
apparatus (????). Thus, the lifetime measurement desribed in this report is
only relevant to muon decays \eqref{mudecay} and \eqref{antimudecay}.

Due to relativistic time dilation, the flux of muons at sea level
remains large at $10^{-2}$ cm$^-2$s$^-1$sr$^-1$. Since the particles are travelling close to the 


The decay of the muon is desribed by an exponential function

\begin{equation} N(t) = N_0 e^{-\Gamma_{\mu} t} \end{equation} 
here $\Gamma_{\mu}$ is the decay rate, which gives the decay lifetime
$\tau_{\mu} = 1/\Gamma_{\mu}$.

The decay rate $\Gamma_{\mu}$ is proportional to the square of the
amplitude of the decay diagram (Figure ~\ref), which depends on the
product of the couplings at each vertex. In this case, the coupling at
each of the two vertices is proportional to $\sqrt{G_F}$, so we have

\begin{equation}\Gamma_{\mu} \propto G_F^2 \end{equation}

A more involved calculation gives that the lifetime of the muon is 

\begin{equation}\tau_{\mu} = \dfrac{192\pi^3\hbar^7}{G_F^2m_{\mu}^5c^4}\end{equation}
where $c$ is the speed of light, $\hbar$ is Planck's constant, and
$m_{\mu}$ is the rest mass of the muon.

The weak decay of the muon is the clearest of all weak interaction
phenomena in both its experimental and theoretical aspects. Thus, the
muon decay is an effective means of studying the weak force, and
specifically finding the weak coupling constant $g_w$.


\begin{equation}\end{equation}
