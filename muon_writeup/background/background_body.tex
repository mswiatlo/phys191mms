%Background body
%Created MB 03-01

\section{Background}\label{background}

The muon is a lepton, originally discovered by Carl Anderson in 1936
\cite{anderson}. Muons ($\mu^-$) and antimuons($\mu^+$) are the most
numerous charged particles at sea level\cite{pdg}. Most of them are
produced at a height of about $15$ km in decays of charged kaons and
pions which are formed via the interaction of cosmic ray particles
with the Earth's upper atmosphere \cite{amsler}. The muon is produced
in a weak decays as shown for example in Figure~\ref{figure:pidecay}
and at sea level forms $80\%$ of cosmic ray flux \cite[p.~8]{rossi}.

\begin{figure}[h]
\begin{center}
\fcolorbox{white}{white}{
  \begin{picture}(352,118) (131,-75)
%    \caption{blahblahblah}
    \SetWidth{1.0} \SetColor{Black}
    \Line[arrow,arrowpos=0.5,arrowlength=5,arrowwidth=2,arrowinset=0.2](160,22)(256,-26)
    \Line[arrow,arrowpos=0.5,arrowlength=5,arrowwidth=2,arrowinset=0.2](256,-26)(160,-74)
    \Line[dash,dashsize=10,arrow,arrowpos=0.5,arrowlength=5,arrowwidth=2,arrowinset=0.2](256,-26)(352,-26)
    \Line[arrow,arrowpos=0.5,arrowlength=5,arrowwidth=2,arrowinset=0.2](352,-26)(432,22)
    \Line[arrow,arrowpos=0.5,arrowlength=5,arrowwidth=2,arrowinset=0.2](432,-74)(352,-26)
    \Text(144,22)[lb]{\large{\Black{$d$}}}
    \Text(144,-74)[lb]{\large{\Black{$u$}}}
    \Text(128,-26)[lb]{\large{\Black{$\pi^{-}$}}}
    \Text(448,22)[lb]{\large{\Black{$\mu^{-}$}}}
    \Text(448,-74)[lb]{\large{\Black{$\nu_{\mu}$}}}
    \Text(304,-10)[lb]{\large{\Black{$W^{-}$}}}
  \end{picture}
}
\caption{Weak decay of $\pi^-$, producing $\mu^-$.}
\label{figure:pidecay}
\end{center}
\end{figure}


\subsection{Muon Decay}

In free space, negatively charged muons decay weakly into an electron,
muon neutrino, and electron antineutrino \cite{easwar}
(Figure~\ref{figure:mudecay}):

\begin{equation}\mu^- \rightarrow e^-~\nu_{\mu}~\overline{\nu_e}\label{mudecay}\end{equation}

with a corresponding antimatter process of

\begin{equation}\mu^+ \rightarrow e^+~\overline{\nu_{\mu}}~{\nu_e}\label{antimudecay}\end{equation}

\begin{figure}[h]
\begin{center}
\fcolorbox{white}{white}{
  \begin{picture}(336,166) (131,-75)
    \SetWidth{1.0}
    \SetColor{Black}
    \Line[arrow,arrowpos=0.5,arrowlength=5,arrowwidth=2,arrowinset=0.2](160,6)(304,6)
    \Line[arrow,arrowpos=0.5,arrowlength=5,arrowwidth=2,arrowinset=0.2](304,6)(416,70)
    \Line[dash,dashsize=10,arrow,arrowpos=0.5,arrowlength=5,arrowwidth=2,arrowinset=0.2](304,6)(368,-42)
    \Line[arrow,arrowpos=0.5,arrowlength=5,arrowwidth=2,arrowinset=0.2](368,-42)(416,-10)
    \Line[arrow,arrowpos=0.5,arrowlength=5,arrowwidth=2,arrowinset=0.2](416,-74)(368,-42)
    \Text(128,6)[lb]{\large{\Black{$\mu^{-}$}}}
    \Text(432,-10)[lb]{\large{\Black{$e^{-}$}}}
    \Text(432,-74)[lb]{\large{\Black{$\nu_{e}$}}}
    \Text(432,70)[lb]{\large{\Black{$\nu_{\mu}$}}}
    \Text(304,-42)[lb]{\large{\Black{$W^{-}$}}}
  \end{picture}
}
\caption{Weak decay of $\mu^-$.}
\label{figure:mudecay}
\end{center}
\end{figure}

The decay of the muon is desribed by an exponential function

\begin{equation} N(t) = N_0 e^{-\Gamma_{\mu} t} \label{expdecay}\end{equation} 
here $\Gamma_{\mu}$ is the decay rate, which gives the decay lifetime
$\tau_{\mu} = 1/\Gamma_{\mu}$.

In this experiment, we measure the time between the start event, when
a $\mu^- (\mu^+)$ comes to rest in a scintillator in the lab and the
stop event, which signals the emission of $e^- (e^+)$ in the muon
decay. The histogram of the recorded times is then fit to
\eqref{expdecay} to give the lifetime of the muon.

\subsubsection{Decay in Matter}

The muon lifetime is approximately $2.2 \mu$s\cite{easwar}, second
only to the lifetime of the neutron. In matter, another decay is
possible for $\mu^-$ via nucleus capture:

\begin{equation}\mu^-p^+ \rightarrow n \nu_{\mu} \label{pcap} \end{equation}

Due to the relatively faster process of decay via proton capture, the
mean lifetime of $\mu^-$ in matter is shortened and depends on the
material. The likelyhood of the capture is proportional to $Z^4$,
where $Z$ is the atomic number of the material; for light elements,
the effect of this process is minimal \cite[p.172]{rossi}. For carbon,
for instance, $Z=6$ and the mean lifetime of the muon is theoretically
predicted to be between $1.5$ and $1.9\mu$s \cite[p.~170]{rossi}. 

In addition, the products of the decay\eqref{pcap} are a neutron and a
neutrino. In our experiment, the efficiency of the scintillator in
detecting neutrons is lower than the electrons and positrons which are
results of \eqref{mudecay} and \eqref{antimudecay}, because neutrons
carry no charge. Thus, we expect that the effect of muon capture to be
small, although it may lower the measured lifetime from the
experimentally established value in free space.

\subsection{Effects of Relativistic Time Dilation}

Even with velocities within a percent of the speed of light, the
travel time of the muon from the point of creation in the atmosphere
takes approximately $50\mu$s - over 20 decay lifetimes - to reach the
ground. According to Newtonian physics, the flux would be reduced by a
factor of over $10^{10}$. However, the flux of muons at sea level,
where the lab is located, remains large at $10^{-2}$
cm$^{-2}$s$^{-1}$sr$^{-1}$, only reduced by a factor of $5$ from the peak
flux at $15$km \cite{rossi}.

This effect is due to the relativistic time dilation predicted by
Special Relativity. While in the frame of the laboratory, the time of
flight of the muons is $50\mu$s, the muon itself experiences a proper
time reduced by a factor of $\gamma$: $ \gamma t_{\mu} = t_{lab}$,
where $\gamma = 1/\sqrt{1 - \frac{v_{\mu}^2}{c^2}}$. Since the particles are
travelling close to the speed of light, the relativistic correction
becomes non-negligible. With muon speeds ranging from $.994c$ to
$.998c$, the proper time experienced by the muon is between $3.2$ and
$5.5\mu$s, less than $2$ lifetimes on average. 

The time in flight is still on the same order, even greater than, the
lifetime of muon decay which our experiment seeks to
measure. Nevertheless, the time the muons experience in the atmosphere
prior to stopping in the detector has no effect on the decay rate
measurement. While we do sample fewer short decay times and slow
moving muons, this fact simply decreases the amount of data without
affecting the parameters of the exponential.

\subsection{Muon Mass}

The experimental setup which records the time between a muon stopping
event and a released electron or positron event needs to be only
slightly modified in order to measure the mass of the muon.

In order to measure muon mass, we consider the products of the $\mu^-$
decay: electron and two neutrinos \eqref{mudecay}; the antimatter
decay analysis is equivalent to the following. For a muon that is
stopped in the scintillator, the center of mass frame is the same as
the lab frame. Then, since the electron mass is only $0.5\%$ of the
muon mass and the neutrinos are essentially massless, to a good
approximation we can assume that the rest energy of the muon is fully
converted to the kinetic energy of the $e^-$ and neutrinos. Then,
measuring the energy distribution of the emitted electrons will
provide information regarding the initial muon mass. Specifically, due
to conservation of momentum, the magnitude of the electron momentum,
$p_e$, must equal the sum of the neutrino momenta. Fixing $p_e$ along
the x-axis, the only possible scenario of the decay is pictured in
Figure ~\ref, where $0 \leq \theta <\pi/2$ and is measured from the
negative x-axis.



\begin{figure}[htbp]
\begin{center}
 
\input{./figures/electron_neutrino.pstex_t}
 
\caption{\small{Decay products of the muon. Due to conservation of
energy and momentum, the momenta of the two neutrinos but be at equal
angles, $\theta$ and $-\theta$, from the direction of the electron
momentum. Maximizing the electron momentum (and therefore energy) is
the configuration with $\theta = 0$.}}
\label{e_nu}
\end{center}
\end{figure}


The total momentum of the neutrinos is minimized when there is no y
component, that is $\theta = 0$. In this case, $p_e = p_e^{max} =
\frac{1}{2}p_{tot}$. Again neglecting electron mass, we have $E_e =
p_ec$, so the energy of the electron is maximized when the momentum is
maximized, and so

\begin{equation}E_e^{max} = \frac{1}{2}E_{\mu} = \frac{1}{2}m_{\mu}\end{equation}

Then by measuring the electron energy spectrum, which is a $\beta$
decay spectrum with a cutoff at $ \frac{1}{2}m_{\mu}$, we can find the
maximum electron energy and thus measure the muon mass.

\subsubsection{Energy Calibration}

Our intruments enable us to measure the height of the pulse from the
electron that registers on the PMT. In order to convert the pulse
height distribution to an energy distribution, we have to calibrate
the pulse height voltage in terms of energy left in the detector.

To do so, we find the pulse height voltage which corresponds to the
minimum ionization energy (approximately $2$ MeVg$^{-1}$cm$^{2}$) by
finding the maximum in the peak height distribution of muons which
pass through all three scintillators. Since the incoming muons have a
relatively random distribution of momenta, the most frequent rate of
energy loss will be near a local extremum, which in this case is a
minimum (see appendix blah for more details).

Assuming the scintillator light output varies linearly with the amount
of energy deposited, and measuring the scintillator density and
thickness, we find a conversion ratio between pulse voltage and
electron energy.

\subsection{Weak Force Coupling Constant}

The decay rate $\Gamma_{\mu}$ is proportional to the square of the
amplitude of the decay diagram (Figure ~\ref), which depends on the
product of the couplings at each vertex. In this case, the coupling at
each of the two vertices is proportional to $\sqrt{G_F}$, the Fermi
coupling constant, so we have

\begin{equation}\Gamma_{\mu} \propto G_F^2 \end{equation}

A more involved calculation \cite[p.~310-314]{griffiths} gives that the
lifetime of the muon is

\begin{equation}\tau_{\mu} = \dfrac{192\pi^3\hbar^7}{G_F^2m_{\mu}^5c^4}\end{equation}
where $c$ is the speed of light, $\hbar$ is Planck's constant, and
$m_{\mu}$ is the rest mass of the muon. 

Once we establish the value of $m_{\mu}$, we can find the Fermi
coupling constant $G_F$, which describes the strength of the weak
interaction\footnote{Although $G_F$ is not equivalent to the weak
coupling constant, $g_w$, they are related by the equation
\[G_F\equiv \frac{\sqrt{2}}{8}\left(\frac{g_w}{M_Wc^2}\right)^2(\hbar c)^3\]
where $M_W$ is the mass of the $W$ bosons which mediate the weak
interaction. Thus $G_F$ is sufficient and is commonly used to describe
weak interaction formulas\cite[p.~313]{griffiths}}.

The weak decay of the muon is the clearest of all weak interaction
phenomena in both its experimental and theoretical aspects. Thus, the
muon decay is an effective means of studying the weak force, and
specifically finding the Fermi coupling constant $G_F$.
