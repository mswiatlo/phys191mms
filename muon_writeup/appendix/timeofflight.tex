\section{Muon and Electron Time of Flight}
\label{timeofflight}

Both the muon and electron time of flight between detectors has no impact on the experiment, but for different reasons.

There is no minimum for the energy of the outgoing electron in the muon decay. Consider the decay as drawn in Figure~\ref{figure:electronpi}: momentum conservation gives $\theta < \pi/2$, but momentum conservation also dictates that as $\theta$ approaches $\pi/2$, the outgoing electron's momentum (and therefore energy) become lower and lower, approaching $0$. Thus, the actual time of flight for electrons between detectors is very widely spread. However, the middle discriminator outputs only a $50~\mathrm{ns}$ pulse; to be registered as a proper stop signal, the electron must hit the other detector within this window (which is reduced by the comparison time of the coincidence unit). Thus, the highest possible travel time for the electron is on the order of $50~\mathrm{ns}$, adding a possible error factor of $.05~\mu\mathrm{s}$ to each measurement. Given the relevant timescales on the order of several $\mu\mathrm{s}$, the time of flight becomes completely negligible.

\begin{figure}[htbp]
\begin{center}
 
\input{./figures/electron_neutrino_bigth2.pstex_t}
 
\caption{\small{Decay products of the muon, with high $\theta$ and low $E_{e}$}}
\label{figure:electronpi}
\end{center}
\end{figure}

However, these considerations do not save the muon lifetime calculation; consider the scenario where a muon moves arbitrarily slowly between the middle and bottom detectors (slow enough to escape the $50 \mathrm{ns}$ middle pulse). Then the coincidence unit would detect the event as $T \wedge M \wedge \bar{B}$, even though it was really a $T \wedge M \wedge B$, and therefore a false start signal is created.

But from the discussion in Appendix~\ref{masscalibration}, it is clear that at least on the order of $10 \mathrm{MeV}$ are deposited by a muon as it passes through the detector. The percentage of muons that have just over $10~\mathrm{MeV}$ in kinetic energy and deposit only $10 \mathrm{MeV}$ is vanishingly small, as it corresponds to a very small range of velocities and only the minimum deposited energy. Thus, most muons leaving a detector will have at least on the order of, say, $207 \mathrm{MeV}$ of energy, where $206 \mathrm{MeV}$ is the rest energy and there is approximately $1 \mathrm{MeV}$ of kinetic energy. As total energy is given by $E = \gamma m$, this corresponds to $\gamma = 1.0045$, and $\beta = .10$: i.e., even the muons on the slower end of the spectrum have velocities about 1/10th the speed of light. As the distance between detectors is on the order of $2 \mathrm{cm}$, the time of flight is approximately $700 \mathrm{ps}$, which is completely negligible. Even muons with less energy would still move between detectors quickly enough to not affect our experiment. Very few will actually move slowly enough to outlast the $50$ ns pulse; these will contribute to false starts and noise, but not significantly. 