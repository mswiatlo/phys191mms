\begin{abstract}
We present our measurements of the lifetime and mass of cosmic ray
muons. The detection of muons is performed at sea level using a simple
system of three scintillators and photomultiplier tubes. We find the
average lifetime of the muon to be $\tau{\mu} = 2.12 \pm .16 \mu$s,
which is in good agreement with the accepted value of $2.20 \mu$s
\cite{pdg}. The muon mass is measured to be $m_{\mu} = 120 \pm 20$
MeV/c$^2$, which includes the accepted value of $105.66$ MeV/c$^2$
\cite{pdg}. These measurements are used to calculate the Fermi
coupling constant $G_F$. Our value of $G_F$ is determined to be $3.0
\pm 1.3\times 10^{-59}$ J m$^3$, which is in correspondence with
$4.3\times 10^{-59}$ J m$^3$, the experimentally established value
\cite{pdg}. This experiment allows us to use a low energy setup to
successfully study the weak interaction.


\end{abstract}